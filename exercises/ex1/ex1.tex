%%%%%%%%%%%%%%%%%%
% DOCUMENT CLASS %
%%%%%%%%%%%%%%%%%%
\documentclass{article}

%%%%%%%%%%%%
% PACKAGES %
%%%%%%%%%%%%
\usepackage{hyperref}
\usepackage{listings}

%%%%%%%%%%%%%%%%%%
% BEGIN DOCUMENT %
%%%%%%%%%%%%%%%%%%
\begin{document}

%%%%%%%%%
% TITLE %
%%%%%%%%%
\title{Exercise 1}

%%%%%%%%%%%%%%%%%%%%%%%%%%%%%
% AUTHOR = COURSE NAME HERE %
%%%%%%%%%%%%%%%%%%%%%%%%%%%%%
%\author{Compilation 0368:3133}

%%%%%%%%%%%%%%%%%%%%%%%%%%%%%%%
% DATE = SUBMISSION DATE HERE %
%%%%%%%%%%%%%%%%%%%%%%%%%%%%%%%
\date{Due 3/11/2021, 23:59}

%%%%%%%%%
% TITLE %
%%%%%%%%%
\maketitle

\newcommand{\plname}{\textbf{L}\ }

%%%%%%%%%%%%%%%%%%%%%%%%%%%
% SECTION :: Introduction %
%%%%%%%%%%%%%%%%%%%%%%%%%%%
\section{Introduction}
During the semester we will implement a compiler to an invented
object oriented language called \plname.
In order to make this document self contained,
all the information needed to complete the first exercise is brought here.

%%%%%%%%%%%%%%%%%%%%%%%%%%%%%%%%%%%%%
% SECTION :: Programming Assignment %
%%%%%%%%%%%%%%%%%%%%%%%%%%%%%%%%%%%%%
\section{Programming Assignment}
The first exercise implements a lexical scanner based on the
open source tool \href{http://jflex.de/}{JFlex}.
The input for the scanner is a single text file containing a \plname program,
and the output is a (single) text file containing a tokenized representation of the input.
The course repository contains a simple skeleton program,
and you are encouraged to work your way up from there.

%%%%%%%%%%%%%%%%%%%%%%%%%%%%%%%%%%%%%
% SECTION :: Lexical Considerations %
%%%%%%%%%%%%%%%%%%%%%%%%%%%%%%%%%%%%%
\section{Lexical Considerations}
\paragraph{Identifiers} may contain letters and digits, and must start with a letter.
\paragraph{Keywords} in Table \ref{Table_Reserved_Keywords} can \textit{not}
be used as identifiers.
\begin{table}[h]
\centering
\begin{tabular}{l l l l l}
%%%%%%%%%%%%%%%%%%%%%%%%%%%%%%%%%%%%
class & nil & array & while & int \\
extends & return & new & if & string \\
%%%%%%%%%%%%%%%%%%%%%%%%%%%%%%%%%%%%
\end{tabular}
\caption{
Reserved keywords of \plname.
\label{Table_Reserved_Keywords}}
\end{table}
\paragraph{White spaces} consist of spaces, tabs and newlines characters.
They may appear between any tokens.
\paragraph{Comments} in \plname are similar to those used in the C programming language,
but may only contain characters from Table \ref{Table_Characters_That_May_Appear_inside_Comments}.
\\
\textit{Type-1 Comment}: \\
A sequence of characters that begins with // followed by any character from Table \ref{Table_Characters_That_May_Appear_inside_Comments}
up to the first occurrence of a newline character.
Note, for example, that //a\$ is not considered as a \textit{Type-1} comment.
\\
\textit{Type-2 Comment}: \\
A sequence of characters that begins with /* followed by any characters from Table \ref{Table_Characters_That_May_Appear_inside_Comments},
up to the first occurrence of the end sequence */.
An unclosed \textit{Type-2} comments (that is, a missing */) is a lexical error.
For example, the input /*a is a lexical error, but a*/ is valid.
\\
\\
Comments (of both types) that contain characters that do not appear in Table \ref{Table_Characters_That_May_Appear_inside_Comments} are lexical errors.

\begin{table}[h]
\centering
\begin{tabular}{ | c | c | c | c | }
\hline
%%%%%%%%%%%%%%%%%%%%%%%%%%%%%%%%%%%%%%%%%%%%%%%%%%%%%
        &        &              &                  \\
letters & digits & white spaces & line terminators \\
        &        &              &                  \\
%%%%%%%%%%%%%%%%%%%%%%%%%%%%%%%%%%%%%%%%%%%%%%%%%%%%%
\hline
%%%%%%%%%%%%%%%%%%%%%%%%%%%%%%%%%%%%%%%%%%%%%%%%%%%%%%%%%%%%%%%%%%%%
                  &     &                 &                       \\
( ) [ ] $\{$ $\}$ & ? ! & $+$ $-$ $*$ $/$ & . (dot) ; (semicolon) \\
                  &     &                 &                       \\
%%%%%%%%%%%%%%%%%%%%%%%%%%%%%%%%%%%%%%%%%%%%%%%%%%%%%%%%%%%%%%%%%%%%
\hline
\end{tabular}
\caption{
Characters that may appear inside comments in \plname.
\label{Table_Characters_That_May_Appear_inside_Comments}}
\end{table}

\paragraph{Integers} in \plname are sequences of digits.
%literals may start with an optional negation sign $-$, followed by a sequence of digits.
Integers should \textit{not} have leading zeroes,
and when they do, it is a lexical error.
%In addition, $-0$ is a lexical error.
Though integers are stored as 32 bits in memory,
they are artificially limited in \plname to have
16-bits signed values between $0$ and $2^{15}-1$.
Integers out of this range are lexical errors.
\paragraph{Strings} are sequences of (zero or more) letters between double quotes.
Strings that contain non letter characters are lexical errors.
Unclosed strings are lexical errors too.
\section{Input}
The input for this exercise is a single text file, the input \plname program.
%%%%%%%%%%%%%%%%%%%%%
% SECTION :: Output %
%%%%%%%%%%%%%%%%%%%%%
\section{Output}
The output is a single text file that contains a tokenized representation of the input program.
Each token should appear in a separate line, together with the line number
it appeared on, and the character position inside that line.
The list of token names appears in Table \ref{Table_Token_Names_For_Exercise_1},
and will only be used in this first exercise.
Later phases of the compiler will make no use of these token names.
\begin{table}[h]
\centering
\begin{tabular}{|l|c|c|l|c|}
\hline
Token Name & Description & & Token Name & Description \\
\hline
\hline
%%%%%%%%%%%%%%%%%%%%%%%%%%%%%%%%%%%%%%%%%%%%%%%%%%%%%%%%%%%%%%%%%%%
LPAREN    & $($  & & ASSIGN           & $:=$                     \\
RPAREN    & $)$  & & EQ               & $=$                      \\
LBRACK    & $[$  & & LT               & $<$                      \\
RBRACK    & $]$  & & GT               & $>$                      \\
LBRACE    & $\{$ & & ARRAY            & array                    \\
RBRACE    & $\}$ & & CLASS            & class                    \\
NIL       & nil  & & EXTENDS          & extends                  \\
PLUS      & $+$  & & RETURN           & return                   \\
MINUS     & $-$  & & WHILE            & while                    \\
TIMES     & $*$  & & IF               & if                       \\
DIVIDE    & $/$  & & NEW              & new                      \\
COMMA     & $,$  & & INT$(value)$     & $value$ is an integer    \\
DOT       & $.$  & & STRING$(value)$  & $value$ is a string      \\
SEMICOLON & $;$  & & ID$(value)$      & $value$ is an identifier \\
TYPE\_INT & $int$  & & TYPE\_STRING   & string \\
TYPE\_VOID & $void$  & & \\
%%%%%%%%%%%%%%%%%%%%%%%%%%%%%%%%%%%%%%%%%%%%%%%%%%%%%%%%%%%%%%%%%%%
\hline
\end{tabular}
\caption{
Token names for the first exercise.
Note that three types of tokens are associated with corresponding values:
integers, identifiers and strings.
The rest of the tokens encountered only contain their name.
\label{Table_Token_Names_For_Exercise_1}}
\end{table}
Three types of tokens are associated with corresponding values: integers, identifiers and strings.
The printing format for these tokens can be easily deduced from the examples in Table
\ref{Table_Token_Printing_Examples_For_Exercise_1}.
\begin{table}[h]
\centering
\begin{tabular}{|l|l| }
Printed Lines Examples & Description \\
\hline
\hline
%%%%%%%%%%%%%%%%%%%%%%%%%%%%%%%%%%%%%%%%%%%%%%%%%%%%%%%%%%%%%%%%%%%%%%%%%%%%%%%%%%%%%%%%
LPAREN[$7$,$8$] & left parenthesis is encountered in line $7$, character position $8$ \\
%%%%%%%%%%%%%%%%%%%%%%%%%%%%%%%%%%%%%%%%%%%%%%%%%%%%%%%%%%%%%%%%%%%%%%%%%%%%%%%%%%%%%%%%
\hline
%%%%%%%%%%%%%%%%%%%%%%%%%%%%%%%%%%%%%%%%%%%%%%%%%%%%%%%%%%%%%%%%%%%%%%%%%%%%%%%%%%%%%%%
INT($74$)[$3$,$8$] & integer $74$ is encountered in line $3$, character position $8$ \\
%%%%%%%%%%%%%%%%%%%%%%%%%%%%%%%%%%%%%%%%%%%%%%%%%%%%%%%%%%%%%%%%%%%%%%%%%%%%%%%%%%%%%%%
\hline
%%%%%%%%%%%%%%%%%%%%%%%%%%%%%%%%%%%%%%%%%%%%%%%%%%%%%%%%%%%%%%%%%%%%%%%%%%%%%%%%%%%%%%%%%%%%%
STRING(``Dan")[$2$,$5$] & string ``Dan" is encountered in line $2$, character position $5$ \\
%%%%%%%%%%%%%%%%%%%%%%%%%%%%%%%%%%%%%%%%%%%%%%%%%%%%%%%%%%%%%%%%%%%%%%%%%%%%%%%%%%%%%%%%%%%%%
\hline
%%%%%%%%%%%%%%%%%%%%%%%%%%%%%%%%%%%%%%%%%%%%%%%%%%%%%%%%%%%%%%%%%%%%%%%%%%%%%%%%%%%%%%%%%%%%%
ID(numPts)[$1$,$6$] & identifier numPts is encountered in line $1$, character position $6$ \\
%%%%%%%%%%%%%%%%%%%%%%%%%%%%%%%%%%%%%%%%%%%%%%%%%%%%%%%%%%%%%%%%%%%%%%%%%%%%%%%%%%%%%%%%%%%%%
\hline
\end{tabular}
\caption{
Printing format for the first exercise.
Each line in the output text file should contain the token name,
the line number they appeared on, and the character position inside that line.
Note that three types of tokens are associated with corresponding values:
integers, identifiers and strings.
These values should appear inside the parentheses as shown.
\label{Table_Token_Printing_Examples_For_Exercise_1}}
\end{table}
Whenever the input program contains a lexical error, the output file
should contain a \textit{single} word only: ERROR.
%%%%%%%%%%%%%%%%%%%%%%%%%%%%%%%%%%%%
% SECTION :: Submission Guidelines %
%%%%%%%%%%%%%%%%%%%%%%%%%%%%%%%%%%%%
\section{Submission Guidelines}
\subsection{Project Repository}
Create an account on \href{https://github.com/}{GitHub}. \\
Then, visit \href{https://education.github.com/pack}{this page} to enable free creation of private repositories. \\
One team member should create a new \textit{private} repository called \textit{compilation},
and then invite other team members, the course grader (omarmahamid) and myself (davidtr1037) as collaborators. \\
Please put inside the uppermost folder (\textit{compilation}) the following text files:
\begin{itemize}
    \item \textit{ids.txt:} the ID's of all team members (one ID per line).
    \item \textit{users.txt:} the GitHub user names of all team members (one user name per line).
    \item \textit{names.txt:} the \textit{hebrew} names of all team members (one name per line).
\end{itemize}

In addition, \textit{compilation} should contain a sub folder called \textit{ex1} where your code will reside.
\textit{compilation/ex1} should contain a makefile building your source files to a
runnable jar file called \textit{LEXER} (note the lack of the .jar suffix).
Feel free to use the makefile supplied in the course repository,
or write a new one if you want to. 
Before you submit, make sure that your exercise compiles and runs
on the school server: $nova.cs.tau.ac.il$.
This is the formal running environment of the course.

\subsection{Command-line usage}
\textit{LEXER} receives $2$ parameters (file paths):
\begin{itemize}
    \item \textit{input} (input file path)
    \item \textit{output} (output file path containing the expected output)
\end{itemize}

\subsection{Skeleton}
You are encouraged to use the makefile provided by the exercise skeleton,
which can be found in the course repository using the following link: \\
\url{https://github.com/davidtr1037/compilation-tau/tree/master/src/ex1}
Some other files of intereset in the provided skeleton:
\begin{itemize}
    \item \textit{jflex/LEX\_FILE.lex} (LEX configuration file)
    \item \textit{src/TokenNames.java} (enum with token types)
    \item \textit{src/Main.java}
\end{itemize}
To use the skeleton, run the following command (in the \textit{src/ex1} directory): \\
\texttt{\$ make} \\
This performs the following steps:
\begin{itemize}
    \item Generates \textit{src/Lexer.java}
    \item Compiles the modules into \textit{LEXER}
    \item Runs \textit{LEXER} on \textit{input/Input.txt}
\end{itemize}

\subsection{Environment requirements}
\begin{itemize}
\item Ubuntu 14.04 / 16.04
\item Java 8
\end{itemize}

\subsection{GitHub related material}
\begin{itemize}
    \item \url{https://guides.github.com/activities/hello-world/}
    \item \url{https://docs.github.com/en/get-started}
\end{itemize}

\end{document}
