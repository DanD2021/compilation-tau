%%%%%%%%%%%%%%%%%%
% DOCUMENT CLASS %
%%%%%%%%%%%%%%%%%%
\documentclass{article}

%%%%%%%%%%%%
% PACKAGES %
%%%%%%%%%%%%
\usepackage{hyperref}

%%%%%%%%%%%%%%%%%%
% BEGIN DOCUMENT %
%%%%%%%%%%%%%%%%%%
\begin{document}

%%%%%%%%%
% TITLE %
%%%%%%%%%
\title{Exercise 2}

%%%%%%%%%%%%%%%%%%%%%%%%%%%%%
% AUTHOR = COURSE NAME HERE %
%%%%%%%%%%%%%%%%%%%%%%%%%%%%%
%\author{Compilation 0368:3133}

%%%%%%%%%%%%%%%%%%%%%%%%%%%%%%%
% DATE = SUBMISSION DATE HERE %
%%%%%%%%%%%%%%%%%%%%%%%%%%%%%%%
\date{Due 17/11/2021, before 23:59}

%%%%%%%%%
% TITLE %
%%%%%%%%%
\maketitle

\newcommand{\plname}{\textbf{L}\ }

%%%%%%%%%%%%%%%%%%%%%%%%%%%
% SECTION :: Introduction %
%%%%%%%%%%%%%%%%%%%%%%%%%%%
\section{Introduction}
We continue our journey of building a compiler
for the invented object oriented language \plname.
In order to make this document self contained,
all the information needed to complete the second exercise is brought here.
%%%%%%%%%%%%%%%%%%%%%%%%%%%%%%%%%%%%%
% SECTION :: Programming Assignment %
%%%%%%%%%%%%%%%%%%%%%%%%%%%%%%%%%%%%%
\section{Programming Assignment}
The second exercise implements a \href{http://www2.cs.tum.edu/projects/cup/}{CUP} based
parser on top of your \href{http://jflex.de/}{JFlex} scanner from the exercise $1$.
The input for the parser is a single text file containing a \plname program,
and the output is a (single) text file indicating whether the input program
is syntactically valid or not. In addition to that,
whenever the input program has correct syntax,
the parser should internally create the abstract syntax tree (AST).
Currently, the course repository contains a simple skeleton
parser, that indicates whether the input program has correct syntax,
and internally builds an AST for a small subset of \plname.
As always, you are encouraged to work your way up from there,
but feel free to write the whole exercise from scratch if you want to.
Note also, that the AST will not be checked in exercise $2$.
It is needed for later phases (semantic analyzer and code generation)
but the best time to design and implement the AST is exercise $2$.
%%%%%%%%%%%%%%%%%%%%%%%%%%%%%%%%%
% SECTION :: The Syntax %
%%%%%%%%%%%%%%%%%%%%%%%%%%%%%%%%%
\section{The \plname Syntax}
Table \ref{Table_CFG} specifies the context free grammar of \plname.
You will need to feed this grammar to \href{http://www2.cs.tum.edu/projects/cup/}{CUP},
and make sure there are no shift-reduce conflicts.
The operator precedence is listed in Table
\ref{Table_Binary_Operators}. 
\begin{table}[h]
\centering
\begin{tabular}{ l c l }
%%%%%%%%%%%%%%%%%%%%%%%%%%%%%%%
Program  & $::=$ & dec$^{+}$ \\
%%%%%%%%%%%%%%%%%%%%%%%%%%%%%%%
\\
%%%%%%%%%%%%%%%%%%%%%%%%%%%%%%%%%%%%%%%%%%%%%%%%%%%%%%%%%%%%%%%%%%
dec      & $::=$ & varDec $|$ funcDec $|$ classDec $|$ arrayDec \\
%%%%%%%%%%%%%%%%%%%%%%%%%%%%%%%%%%%%%%%%%%%%%%%%%%%%%%%%%%%%%%%%%%
\\
%%%%%%%%%%%%%%%%%%%%%%%%%%%%%%%%%%%%%%%%%%%%%%%%%%%%%%%%%%%%%%%%%%
type      & $::=$ & TYPE\_INT $|$ TYPE\_STRING $|$ ID \\
%%%%%%%%%%%%%%%%%%%%%%%%%%%%%%%%%%%%%%%%%%%%%%%%%%%%%%%%%%%%%%%%%%
\\
%%%%%%%%%%%%%%%%%%%%%%%%%%%%%%%%%%%%%%%%%%%%%%%%%%
varDec   & $::=$ & type ID $[$ ASSIGN exp $]$ ';' \\
         & $::=$ & type ID ASSIGN newExp ';'      \\
%%%%%%%%%%%%%%%%%%%%%%%%%%%%%%%%%%%%%%%%%%%%%%%%%%
%%%%%%%%%%%%%%%%%%%%%%%%%%%%%%%%%%%%%%%%%%%%%%%%%%%%%%%%%%%%%%%%%%%%%%%%%%%
funcDec  & $::=$ & type ID $'('$ $[$ type ID $[$ ',' type ID $]^{*}$ $]$ $')'$ %%
                   $'\{'$ stmt   $[$ stmt $]^{*}$ $'\}'$                 \\
%%%%%%%%%%%%%%%%%%%%%%%%%%%%%%%%%%%%%%%%%%%%%%%%%%%%%%%%%%%%%%%%%%%%%%%%%%%
%%%%%%%%%%%%%%%%%%%%%%%%%%%%%%%%%%%%%%%%%%%%%%%%%%%%%%%%%%%%%%%%%%%%%%%%%%%%%%%%%%%%%%%%%
classDec & $::=$ & CLASS ID $[$ EXTENDS ID $]$ $'\{'$ cField $[$ cField $]^{*}$ $'\}'$ \\
%%%%%%%%%%%%%%%%%%%%%%%%%%%%%%%%%%%%%%%%%%%%%%%%%%%%%%%%%%%%%%%%%%%%%%%%%%%%%%%%%%%%%%%%%
%%%%%%%%%%%%%%%%%%%%%%%%%%%%%%%%%%%%%%%%%%%%%%%%%
arrayDec & $::=$ & ARRAY ID $=$ type $'['$ $']'$ \\
%%%%%%%%%%%%%%%%%%%%%%%%%%%%%%%%%%%%%%%%%%%%%%%%%
\\
%%%%%%%%%%%%%%%%%%%%%%%%%%%%%%%%%%%%%%%%%%%%%%%%%%%%%%%%%%%%%%%%%%%%%%%%%%%%%%%%%%%%
exp      & $::=$ & var                                                            \\
         & $::=$ & $'('$ exp $')'$                                                \\
         & $::=$ & exp BINOP exp                                                  \\
         & $::=$ & $[$ var '.' $]$ ID $'('$ $[$ exp $[$ ',' exp $]^{*}$ $]$ $')'$ \\
         & $::=$ & $['-']$ INT $|$ NIL $|$ STRING                                 \\
%%%%%%%%%%%%%%%%%%%%%%%%%%%%%%%%%%%%%%%%%%%%%%%%%%%%%%%%%%%%%%%%%%%%%%%%%%%%%%%%%%%%
\\
%%%%%%%%%%%%%%%%%%%%%%%%%%%%%%%%%%%%%%%%%
var      & $::=$ & ID                  \\
         & $::=$ & var '.' ID          \\
         & $::=$ & var $'['$ exp $']'$ \\
%%%%%%%%%%%%%%%%%%%%%%%%%%%%%%%%%%%%%%%%%
\\  
%%%%%%%%%%%%%%%%%%%%%%%%%%%%%%%%%%%%%%%%%%%%%%%%%%%%%%%%%%%%%%%%%%%%%%%%%%%%%%%%%%%%%%%%
stmt     & $::=$ & varDec                                                             \\
         & $::=$ & var ASSIGN exp ';'                                                 \\
         & $::=$ & var ASSIGN newExp ';'                                              \\
         & $::=$ & RETURN $[$ exp $]$ ';'                                             \\
         & $::=$ & IF $'('$ exp $')'$ $'\{'$ stmt $[$ stmt $]^{*}$ $'\}'$             \\
         & $::=$ & WHILE $'('$ exp $')'$ $'\{'$ stmt $[$ stmt $]^{*}$ $'\}'$          \\
         & $::=$ & $[$ var '.' $]$ ID $'('$ $[$ exp $[$ ',' exp $]^{*}$ $]$ $')'$ ';' \\
%%%%%%%%%%%%%%%%%%%%%%%%%%%%%%%%%%%%%%%%%%%%%%%%%%%%%%%%%%%%%%%%%%%%%%%%%%%%%%%%%%%%%%%%
\\
%%%%%%%%%%%%%%%%%%%%%%%%%%%%%%%%%%%%%%%%%%%%%%%%%%%%%%%
newExp   & $::=$ & NEW ID $|$ NEW ID $'['$ exp $']'$ \\
%%%%%%%%%%%%%%%%%%%%%%%%%%%%%%%%%%%%%%%%%%%%%%%%%%%%%%%
\\
%%%%%%%%%%%%%%%%%%%%%%%%%%%%%%%%%%%%%%%%
cField   & $::=$ & varDec $|$ funcDec \\
%%%%%%%%%%%%%%%%%%%%%%%%%%%%%%%%%%%%%%%%
%%%%%%%%%%%%%%%%%%%%%%%%%%%%%%%%%%%%%%%%%%%%%%%%%%%%%%%%%%%%%%%%%%%%%%%%%
BINOP    & $::=$ & $+$ $|$ $-$ $|$ $*$ $|$ $/$ $|$ $<$ $|$ $>$ $|$ $=$ \\
INT      & $::=$ & $[1-9][0-9]^{*}$ $|$ $0$                            \\
%%%%%%%%%%%%%%%%%%%%%%%%%%%%%%%%%%%%%%%%%%%%%%%%%%%%%%%%%%%%%%%%%%%%%%%%%
\\
\end{tabular}
\caption{
Context free grammar for the \plname programming language.
\label{Table_CFG}}
\end{table}
\begin{table}[h]
\centering
\begin{tabular}{ |c|c|l|l| }
\hline
Precedence & Operator & Description & Associativity \\
\hline
\hline
%%%%%%%%%%%%%%%%%%%%%%%%%%%%%%%%%%%%%%%%%%%%%%%%%%%%%%%%
1          & $:=$            & assign         &       \\
%%%%%%%%%%%%%%%%%%%%%%%%%%%%%%%%%%%%%%%%%%%%%%%%%%%%%%%%
\hline
%%%%%%%%%%%%%%%%%%%%%%%%%%%%%%%%%%%%%%%%%%%%%%%%%%%%%%%%
2          & $=$             & equals         & left  \\
%%%%%%%%%%%%%%%%%%%%%%%%%%%%%%%%%%%%%%%%%%%%%%%%%%%%%%%%
\hline
%%%%%%%%%%%%%%%%%%%%%%%%%%%%%%%%%%%%%%%%%%%%%%%%%%%%%%%%
3          & $<,>$           &                & left  \\
%%%%%%%%%%%%%%%%%%%%%%%%%%%%%%%%%%%%%%%%%%%%%%%%%%%%%%%%
\hline
%%%%%%%%%%%%%%%%%%%%%%%%%%%%%%%%%%%%%%%%%%%%%%%%%%%%%%%%
4          & $+,-$           &                & left  \\
%%%%%%%%%%%%%%%%%%%%%%%%%%%%%%%%%%%%%%%%%%%%%%%%%%%%%%%%
\hline
%%%%%%%%%%%%%%%%%%%%%%%%%%%%%%%%%%%%%%%%%%%%%%%%%%%%%%%%
5          & $*,/$           &                & left  \\
%%%%%%%%%%%%%%%%%%%%%%%%%%%%%%%%%%%%%%%%%%%%%%%%%%%%%%%%
\hline
%%%%%%%%%%%%%%%%%%%%%%%%%%%%%%%%%%%%%%%%%%%%%%%%%%%%%%%%
6          & $[$             & array indexing &       \\
%%%%%%%%%%%%%%%%%%%%%%%%%%%%%%%%%%%%%%%%%%%%%%%%%%%%%%%%
\hline
%%%%%%%%%%%%%%%%%%%%%%%%%%%%%%%%%%%%%%%%%%%%%%%%%%%%%%%%
7          & $($             & function call  &       \\
%%%%%%%%%%%%%%%%%%%%%%%%%%%%%%%%%%%%%%%%%%%%%%%%%%%%%%%%
\hline
%%%%%%%%%%%%%%%%%%%%%%%%%%%%%%%%%%%%%%%%%%%%%%%%%%%%%%%%
8          & $.$     & field access   & left          \\
%%%%%%%%%%%%%%%%%%%%%%%%%%%%%%%%%%%%%%%%%%%%%%%%%%%%%%%%
\hline
\end{tabular}
\caption{
Binary operators of \plname along with their associativity and precedence.
$1$ stands for the lowest precedence, and $9$ for the highest.
\label{Table_Binary_Operators}}
\end{table}
%%%%%%%%%%%%%%%%%%%%
% SECTION :: Input %
%%%%%%%%%%%%%%%%%%%%
\section{Input}
The input for this exercise is a single text file, the input \plname program.
%%%%%%%%%%%%%%%%%%%%%
% SECTION :: Output %
%%%%%%%%%%%%%%%%%%%%%
\section{Output}
The output is a \textit{single} text file that contains a \textit{single} word.
Either OK when the input program has correct syntax,
or otherwise ERROR(\textit{location}), where \textit{location}
is the line number of the \textit{first} error that was encountered.

%%%%%%%%%%%%%%%%%%%%%%%%%%%%%%%%%%%%
% SECTION :: Submission Guidelines %
%%%%%%%%%%%%%%%%%%%%%%%%%%%%%%%%%%%%
\section{Submission Guidelines}
The skeleton for this exercise can be found \href{https://github.com/davidtr1037/compilation-tau/tree/master/src/ex2}{here}.
In your project, you need to add the relevant derivation rules and AST constructors.
Your project must contain the makefile in the following path:
\begin{itemize}
    \item ex2/Makefile
\end{itemize}
This makefile should build the parser (a runnable jar file) in the following path:
\begin{itemize}
    \item ex2/PARSER
\end{itemize}
Feel free to reuse the makefile supplied in the skeleton, or write a new one if you want to.

\subsection{Command-line usage}
\textit{PARSER} receives 2 parameters (file paths):
\begin{itemize}
    \item \textit{input} (input file path)
    \item \textit{output} (output file path containing the expected output)
\end{itemize}

\subsection{Skeleton}
You are encouraged to use the makefile provided by the skeleton.
Some files of intereset in the provided skeleton:
\begin{itemize}
    \item \textit{jflex/LEX\_FILE.lex} (LEX configuration file)
    \item \textit{cup/CUP\_FILE.lex} (CUP configuration file)
    \item \textit{src/Main.java}
    \item \textit{src/AST/*.java}
\end{itemize}
To use the skeleton, run the following command (in the \textit{src/ex2} directory): \\
\texttt{\$ make} \\
This performs the following steps:
\begin{itemize}
    \item Generates the relevant files using jflex/cup
    \item Compiles the modules into \textit{PARSER}
    \item Runs \textit{PARSER} on \textit{input/Input.txt}
    \item Generates an image of the resulting syntax tree (for debugging only)
\end{itemize}

\end{document}
