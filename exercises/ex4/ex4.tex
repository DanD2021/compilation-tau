%%%%%%%%%%%%%%%%%%
% DOCUMENT CLASS %
%%%%%%%%%%%%%%%%%%
\documentclass{article}

%%%%%%%%%%%%
% PACKAGES %
%%%%%%%%%%%%
\usepackage{hyperref}
\usepackage{amsmath}
\usepackage{multirow}

%%%%%%%%%%%%%%%%%%
% BEGIN DOCUMENT %
%%%%%%%%%%%%%%%%%%
\begin{document}

%%%%%%%%%
% TITLE %
%%%%%%%%%
\title{Exercise 4}

%%%%%%%%%%%%%%%%%%%%%%%%%%%%%
% AUTHOR = COURSE NAME HERE %
%%%%%%%%%%%%%%%%%%%%%%%%%%%%%
\author{Compilation 0368:3133}

%%%%%%%%%%%%%%%%%%%%%%%%%%%%%%%
% DATE = SUBMISSION DATE HERE %
%%%%%%%%%%%%%%%%%%%%%%%%%%%%%%%
\date{Due 20/1/2022 before 23:55}

%%%%%%%%%
% TITLE %
%%%%%%%%%
\maketitle

\newcommand{\plname}{\textbf{L}\ }

%%%%%%%%%%%%%%%%%%%%%%%%%%%
% SECTION :: Introduction %
%%%%%%%%%%%%%%%%%%%%%%%%%%%
\section{Introduction}
Congratulations, you have made it to the final step of
building an entire compiler for \plname programs.
In order to make this document self contained,
all the information needed to complete the fourth exercise is brought here again.

%%%%%%%%%%%%%%%%%%%%%%%%%%%%%%%%%%%%%
% SECTION :: Programming Assignment %
%%%%%%%%%%%%%%%%%%%%%%%%%%%%%%%%%%%%%
\section{Programming Assignment}
The fourth (and last) exercise implements the code generation phase for \plname programs.
The chosen destination language is \href{https://en.wikipedia.org/wiki/MIPS_architecture}{MIPS}
assembly, favoured for it straightforward syntax,
\href{https://en.wikipedia.org/wiki/SPIM}{complete toolchain}
and available tutorials.
The exercise can be roughly divided into three parts as follows:
$(1)$ recursively traverse the AST to create
an intermediate representation (IR) of the program.
$(2)$ Translate IR to MIPS instructions,
but use an unbounded number of temporaries instead of registers.
$(3)$ Perform liveness analysis, build the interference graph,
and allocate those hundreds (or so) temporaries into $8$ physical registers.
The input for this last exercise is a (single) text file, containing a \plname
program, and the output is a (single) text file that contains the translation
of the input program into MIPS assembly.
%%%%%%%%%%%%%%%%%%%%%%%%%%%%%%%%
% SECTION :: Semantics %
%%%%%%%%%%%%%%%%%%%%%%%%%%%%%%%%
\section{The \plname Semantics}
This section describes the semantics of \plname,
and provides a multitude of example programs.
\newpage
%%%%%%%%%%%%%%%%%%%%%%%%%%%%%%%%%%%%
% SUB-SECTION :: Binary Operations %
%%%%%%%%%%%%%%%%%%%%%%%%%%%%%%%%%%%%
\subsection{Binary Operations}
\paragraph{Integers} in \plname are artificially bounded between $-2^{15}$ and $2^{15}-1$.
The semantics of integer binary operations in \plname is therefore somewhat different
than that of standard programming languages.
It is presented in Table \ref{Table_Semantics_Of_Rio_Mare_Binary_Operations_Between_Integers},
and to distinguish \plname operators from the usual arithmetic signs,
we shall use a \plname subscript inside brackets:
($*_{[\plname]}$, $+_{[\plname]}$ etc.)
\begin{table}[h]
\centering
\begin{tabular}{|p{10cm}|}
\hline
\[
a ~ *_{[\plname]} ~ b ~ = ~
\begin{cases} 
-2^{15}  ~ & ~ ~ ~ ~ ~ ~ ~ ~ \text{when} ~ a*b \in (-\infty, -2^{15}] \\
a*b      ~ & ~ ~ ~ ~ ~ ~ ~ ~ \text{when} ~ a*b \in (-2^{15},2^{15}-1] \\
2^{15}-1 ~ & ~ ~ ~ ~ ~ ~ ~ ~ otherwise                                \\
\end{cases}
\]
\\ \hline
\[
a ~ +_{[\plname]} ~ b ~ = ~
\begin{cases} 
-2^{15}  ~ & ~ ~ ~ ~ ~ ~ ~ ~ \text{when} ~ a+b \in (-\infty, -2^{15}] \\
a+b      ~ & ~ ~ ~ ~ ~ ~ ~ ~ \text{when} ~ a+b \in (-2^{15},2^{15}-1] \\
2^{15}-1 ~ & ~ ~ ~ ~ ~ ~ ~ ~ otherwise                                \\
\end{cases}
\]
\\ \hline
\[
a ~ -_{[\plname]} ~ b ~ = ~
\begin{cases} 
-2^{15}  ~ & ~ ~ ~ ~ ~ ~ ~ ~ \text{when} ~ a-b \in (-\infty, -2^{15}] \\
a-b      ~ & ~ ~ ~ ~ ~ ~ ~ ~ \text{when} ~ a-b \in (-2^{15},2^{15}-1] \\
2^{15}-1 ~ & ~ ~ ~ ~ ~ ~ ~ ~ otherwise                                \\
\end{cases}
\]
\\ \hline
\[
a ~ /_{[\plname]} ~ b ~ = ~
\begin{cases} 
-2^{15}             ~ & ~ ~ ~ ~ ~ ~ ~ ~ \text{when} ~ \lfloor a/b \rfloor \in (-\infty, -2^{15}] \\
\lfloor a/b \rfloor ~ & ~ ~ ~ ~ ~ ~ ~ ~ \text{when} ~ \lfloor a/b \rfloor \in (-2^{15},2^{15}-1] \\
2^{15}-1            ~ & ~ ~ ~ ~ ~ ~ ~ ~ otherwise                                                \\
\end{cases}
\]
\\ \hline
\end{tabular}
\caption{Semantics of \plname binary operations between integers
\label{Table_Semantics_Of_Rio_Mare_Binary_Operations_Between_Integers}}
\end{table}
\paragraph{Strings} can be concatenated with binary operation $+$,
and tested for (contents) equality with binary operator $=$.
When concatenating two (null terminated) strings $\{s_{i}\}_{i=1}^{2}$,
the resulting string $s_{1}s_{2}$ is allocated on the heap,
and should be null terminated. The result of testing contents equality
is either $0$ when they are equal, or $1$ otherwise.
\newpage
%%%%%%%%%%%%%%%%%%%%%%%%%%%%%%%%%%%%%%%%%%
% SUB-SECTION :: If and While Statements %
%%%%%%%%%%%%%%%%%%%%%%%%%%%%%%%%%%%%%%%%%%
\subsection{If and While Statements}
\label{subsection_If_And_While_Statements}
\paragraph{While statements} behave similar to (practically) all programming languages:
before executing their body, their condition is evaluated.
If it equals $0$, the body is ignored, and control is transferred
to the statement immediately after the body.
Otherwise, the body is executed, then the condition is evaluated again, and so forth. 
\paragraph{If statements} behave similar to (practically) all programming languages:
before executing their body, their condition is evaluated.
If it equals $0$, the body is ignored, and control is transferred
to the statement immediately after the body.
Otherwise, the body is executed exactly once,
then control is transferred to the statement immediately after the body. 
%%%%%%%%%%%%%%%%%%%%%%%%%%%%%%%%%%%%%%%%%%%%%%%%%%%%%%%%%
% SUB-SECTION :: Function Calls and Order of Evaluation %
%%%%%%%%%%%%%%%%%%%%%%%%%%%%%%%%%%%%%%%%%%%%%%%%%%%%%%%%%
\subsection{Function Calls and Order of Evaluation}
\label{subsection_Function_Calls_And_Order_Of_Evaluation}
\paragraph{When calling a function} with more than one input parameter,
the evaluation order matters. You should evaluate the sent parameters
from left to right, so for example, the following code should print \verb"32766":
\begin{table}[h]
\centering
\begin{tabular}{|l|}
\hline
%%%%%%%%%%%%%%%%%%%%%%%%%%%%%%%%%%%%%%%%%%%%%%%%%
                                               \\
\verb"class counter { int i := 32767; }"       \\
\verb"counter c := nil;"                       \\
\verb"int inc(){ c.i := c.i + 1; return 0;}"   \\
\verb"int dec(){ c.i := c.i - 1; return 9;}"   \\
\verb"int foo(int m, int n){ return c.i; }"    \\
\verb"void main()"                             \\
\verb"{"                                       \\
\verb"    c := new counter;"                                       \\
\verb"    PrintInt(foo(inc(),dec()));"         \\
\verb"}"                                       \\
                                               \\
%%%%%%%%%%%%%%%%%%%%%%%%%%%%%%%%%%%%%%%%%%%%%%%%%
\hline
\end{tabular}
\caption{Evaluation order of a called function's parameters matters.
\label{Table_Code_Example_Evaluation_Order_Function_Call}}
\end{table}
\paragraph{When evaluating global variables}
\label{Paragraph_When_Evaluating_Global_Variables}
order matters,
and it should be the same as the order of appearance in the original program.
Note that before entering main, all global variables with initialized values
should be evaluated.
\paragraph{When evaluating assignments} order matters,
and the left hand side should be evaluated first.
The same applies for evaluation of \textit{all} binary operations.
\paragraph{Initializing class data members}
should occur during the construction of the object.
The order in which you evaluate the data members initializations is irrelevant.
\newpage
%%%%%%%%%%%%%%%%%%%%%%%%%%%%%%%%%
% SUB-SECTION :: Runtime Checks %
%%%%%%%%%%%%%%%%%%%%%%%%%%%%%%%%%
\subsection{Runtime Checks}
\label{subsection_Runtime_Checks}
\plname enforces three kinds of runtime checks:
division by zero, invalid pointer dereference and out of bound array access.
\paragraph{Division by zero} should be handled by printing ``Division By Zero",
and then exit gracefully by using the exit system call.
The following code will result in such behaviour:
\[
\verb"int i:= 6; while (i+1) { int j := 8/i; i := i-1; }"
\]
\paragraph{Invalid pointer dereference} can occur when trying to
access data members or methods of uninitialized class variable.
For example, here:
\[
\verb"class Father { int i; int j; } Father f; int i := f.i;"
\]
Similarly, assigning \verb"NIL" to \verb"f" should clearly trigger the same behaviour:
\[
\verb"class Father { int i; int j; } Father f := NIL; int i := f.i;"
\]
When an invalid pointer dereference occurs, the program should print
``Invalid Pointer Dereference" and then then exit gracefully by using the exit system call.
\paragraph{Out of bound array access} should be handled by printing "Access Violation" 
and then exit gracefully by using the exit system call.
The following code demonstrates an illegal array access:
\[
\verb"array IntArray = int[]; IntArray A := new int[6]; int i := A[18];"
\]
\newpage
%%%%%%%%%%%%%%%%%%%%%%%%%%%%%%%
% SUB-SECTION :: System Calls %
%%%%%%%%%%%%%%%%%%%%%%%%%%%%%%%
\subsection{System Calls}
\label{subsection_System_Calls}
MIPS supports a limited set of system calls,
out of which we will need only four:
printing an integer,
printing a string,
allocating heap memory and
exit the program.

\begin{table}[h]
\centering
\begin{tabular}{|l|l|l|}
\hline
%%%%%%%%%%%%%%%%%%%%%%%%%%%%%%%%%%%%%%%%%%%%%%%%%
system call example & MIPS code & Remarks      \\
%%%%%%%%%%%%%%%%%%%%%%%%%%%%%%%%%%%%%%%%%%%%%%%%%
\hline
%%%%%%%%%%%%%%%%%%%%%%%%%%%%%%%%%%%%%%%%%%%
\verb"PrintInt(17)" & \verb"li $a0,17" & \\
                    & \verb"li $v0,1"  & \\
                    & \verb"syscall"   & \\
%%%%%%%%%%%%%%%%%%%%%%%%%%%%%%%%%%%%%%%%%%%
\hline
%%%%%%%%%%%%%%%%%%%%%%%%%%%%%%%%%%%%%%%%%%%%%%%%%%%%%%%%%%%%%%%%%%%%%%%%%%%%%%%%%%%%%%%%%%%%%%%%%%%%%%%%
\verb"string s:="``\verb"abc"" & \verb".data"                              & Printed string must be   \\
\verb"PrintString(s)"          & \verb"myLovelyStr: .asciiz "``\verb"abc"" & null terminated.         \\
                               & \verb".text"                              & It can be allocated      \\
                               & \verb"main:"                              & inside the text section, \\
                               & \verb"la $a0,myLovelyStr"                 & or in the heap.          \\
                               & \verb"li $v0,4"                           &                          \\
                               & \verb"syscall"                            &                          \\
%%%%%%%%%%%%%%%%%%%%%%%%%%%%%%%%%%%%%%%%%%%%%%%%%%%%%%%%%%%%%%%%%%%%%%%%%%%%%%%%%%%%%%%%%%%%%%%%%%%%%%%%
\hline
%%%%%%%%%%%%%%%%%%%%%%%%%%%%%%%%%%%%%%%%%%%%%%%%%%%%%%%%%%%%%%%%%%%
\verb"Malloc(17)" & \verb"li $a0,17" & allocated address         \\
                  & \verb"li $v0,9"  & is returned in \verb"$v0" \\
                  & \verb"syscall"   &                           \\
%%%%%%%%%%%%%%%%%%%%%%%%%%%%%%%%%%%%%%%%%%%%%%%%%%%%%%%%%%%%%%%%%%%
\hline
%%%%%%%%%%%%%%%%%%%%%%%%%%%%%%%%%%%%%%%%%%%%%%%%%%%%%%%%%%%%%%%%%%%%%
\verb"Exit()"    & \verb"li $v0,10" & Make sure every              \\
                 & \verb"syscall"   & MIPS program ends with exit. \\
%%%%%%%%%%%%%%%%%%%%%%%%%%%%%%%%%%%%%%%%%%%%%%%%%%%%%%%%%%%%%%%%%%%%%
\hline
\end{tabular}
\caption{Relevant MIPS system calls.
\label{Table_MIPS_System_Calls}}
\end{table}
%%%%%%%%%%%%%%%%%%%%%%%%%%%%%%%
% SUB-SECTION :: System Calls %
%%%%%%%%%%%%%%%%%%%%%%%%%%%%%%%
\subsection{Program entry point (main)}
\label{subsection_Program_Entry_Point_Main}
Every (valid) \plname program has a main function with signature: \verb"void main()".
This function is the entry point of execution (see also \ref{Paragraph_When_Evaluating_Global_Variables}).

%%%%%%%%%%%%%%%%%%%%
% SECTION :: Input %
%%%%%%%%%%%%%%%%%%%%
\section{Input}
The input for this exercise is a single text file, the input \plname program.

%%%%%%%%%%%%%%%%%%%%%
% SECTION :: Output %
%%%%%%%%%%%%%%%%%%%%%
\section{Output}
The output is a single text file that contains the
translation of the input program into MIPS assembly.

\section{Skeleton}
To run the skeleton use the following command:
\begin{itemize}
\item make everything
\end{itemize}
This will create a file with the compiled MIPS code (MIPS.txt), and a file with the output of SPIM (MIPS\_OUTPUT.txt).

\section{SPIM}
We are going to use SPIM 8.0 in the project.
Can be installed using:
\begin{itemize}
\item sudo apt-get install spim xspim
\end{itemize}

%%%%%%%%%%%%%%%%%%%%%%%%%%%%%%%%%%%%
% SECTION :: Submission Guidelines %
%%%%%%%%%%%%%%%%%%%%%%%%%%%%%%%%%%%%
\section{Submission Guidelines}
The skeleton for this exercise can be found \href{https://github.com/davidtr1037/compilation-tau/tree/master/src/ex4}{here}.
The makefile must be located in the following path:
\begin{itemize}
    \item ex4/Makefile
\end{itemize}
This makefile will build the compiler (a runnable jar file) in the following path:
\begin{itemize}
    \item ex4/COMPILER
\end{itemize}
Feel free to reuse the makefile supplied in the skeleton,
or write a new one if you want to.

\subsection{Command-line usage}
\textit{COMPILER} receives 2 parameters (file paths):
\begin{itemize}
    \item \textit{input} (input file path)
    \item \textit{output} (output file path containing the MIPS code)
\end{itemize}

\subsection{Skeleton}
You are encouraged to use the makefile provided by the skeleton.
\subsubsection{Compiling}
To build the skeleton, run the following command (in the \textit{src/ex4} directory): \\
\texttt{\$ make compile} \\
This performs the following steps:
\begin{itemize}
    \item Generates the relevant files using jflex/cup
    \item Compiles the modules into \textit{COMPILER}
\end{itemize}

%The skeleton code for this exercise resides (as usual)
%in subdirectory EX4 of the course repository
%(\url{https://github.com/davidtr1037/COMPILATION_TAU_FOR_STUDENTS/tree/master/FOLDER_3_SOURCE_CODE/EX4}).
%COMPILATION/EX4 should contain a makefile building your source files to a
%runnable jar file called COMPILER (note the lack of the .jar suffix).
%Feel free to use the makefile supplied in the course repository,
%or write a new one if you want to. 
%Before you submit, make sure that your exercise compiles and runs
%on the school server: $nova.cs.tau.ac.il$.
%This is the formal running environment of the course.
%
%\paragraph{Execution parameters}
%compiler receives $2$ input file names:\\ \\
%InputProgram.txt\\
%OutputMIPS.s

\section{Additional Notes}
\subsection{PrintInt}
When printing an integer, print an additional space at the end.

\end{document}
